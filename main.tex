
%Préambule du document :
\documentclass[11pt]{article}

\usepackage[french]{babel}
\usepackage[T1]{fontenc}
\usepackage[utf8]{inputenc}
\usepackage{algorithm}
\usepackage{algpseudocode}
\usepackage{float}
\usepackage{multirow}
\usepackage{tabularx}
\usepackage{hyperref}
\usepackage{amsmath,amsfonts,amssymb}
\usepackage{color}
\usepackage{xcolor}

\title{Math de l'ingénieur}
\author{Florent Gerbaud}
\date{October 2022}

%Corps du document :
\begin{document}
    \maketitle
    \tableofcontents
    \section{Calcul différentiel}
        \subsection{Dérivée}
            \subsubsection{Définition Dérivée}
                \begin{center}
                    Soit $\Omega$ un ouvert, \ 
                    Soit $f: \mathbb{R}^n \longrightarrow \mathbb{R}^m $ \\
                    %$f(x)=\left( a \ b \right)$
                    $f(x)=\begin{pmatrix}
                     f_1(x)\\
                     f_2(x)\\
                    \vdots\\
                     f_m(x)
                    \end{pmatrix}$=$f(x_1,x_2,...,x_m)$
                \end{center}
                On dit que f est dérivable, (ou différentiable) en un point $x\in \Omega$ s'il éxiste une AL $L(x):\mathbb{R}^n \longrightarrow \mathbb{R}^m$ tel que : \\ \\
                $\forall h \in \mathbb{R}^n, x+h \in \Omega$, 
                $f(x+h)=f(x)+L(x).h+o(|h|_{\mathbb{R}^n})$ et tel que $L(x) \in M_{m \times n}$
                \\ la dérivée est L(x)
            \subsubsection{Proposition} 
                \begin{center}
                    
                \end{center}
                f dérivable $\Longrightarrow$ f continue \\
                \color{red} ATTENTION LA RECIPROQUE EST FAUSSE
                \color{black}
            \subsubsection{Théorème dérivation des fonctions composées}
            \begin{align*}
                &Soit \ f:\mathbb{R}^n \longrightarrow \mathbb{R}^m \\
                &Soit \ g:\mathbb{R}^m \longrightarrow \mathbb{R}^p \\
            \end{align*}
            Deux fonctions dérivables. Alors : 
            \begin{align*}
                g \circ f:\mathbb{R}^n \longrightarrow \mathbb{R}^p \ \text{est dérivable et } (g \circ f)'(x)=g'(f(x))f'(x)
            \end{align*}
        \subsection{Dérivée Partielle}
    \section{Espaces vectorielles normés}
\end{document}
